\begin{spacing}{1}
    \chapter*{Abstract}
\end{spacing}
\begin{wrapfigure}{r}{0.3\textwidth}
    \begin{center}
      \includegraphics[width=0.2\textwidth]{pics/wtw-image.png}
    \end{center}
\end{wrapfigure}
In a day to day life, people are often faced with the challenge of
making decisions under uncertainty and time pressure. That 
includes, for example, choosing the outfit of the day based on the weather forecast,
event, ocasion, or personal preferences. In this thesis, we present a novel approach to
assist users in making such decisions by leveraging machine learning techniques.
We developed a mobile application \textbf{(What to wear)} that provides personalized recommendations
based on user preferences, contextual information and needs.
The application uses the tags assigned to clothing items by the user, 
as well as external data sources such as weather forecasts
to generate outfit suggestions.
\newpage
\begin{spacing}{1}
    \chapter*{Zusammenfassung}
\end{spacing}
\begin{wrapfigure}{r}{0.3\textwidth}
    \begin{center}
      \includegraphics[width=0.2\textwidth]{pics/wtw-image.png}
    \end{center}
\end{wrapfigure}
Im Alltag stehen Menschen häufig vor der Herausforderung, 
Entscheidungen unter Unsicherheit und Zeitdruck zu treffen. 
Dazu zählt unter anderem die Wahl des täglichen Outfits basierend 
auf Wettervorhersagen, Anlässen oder persönlichen Vorlieben. 
In dieser Arbeit wird ein neuartiger Ansatz vorgestellt, 
der Nutzer mithilfe von Machine-Learning-Techniken bei solchen Entscheidungen unterstützt.
Zu diesem Zweck wurde die mobile Anwendung What to wear entwickelt,
die personalisierte Outfit-Empfehlungen auf Basis von Nutzerpräferenzen,
Kontextinformationen und individuellen Bedürfnissen bereitstellt. 
Zentrale Grundlage der Empfehlung bilden vom Nutzer vergebene Tags 
zu Kleidungsstücken, die Eigenschaften wie Stil, Anlass oder 
Wettertauglichkeit beschreiben. Ergänzend werden externe 
Datenquellen, insbesondere Wettervorhersagen, herangezogen, 
um passende Outfit-Vorschläge zu generieren.
